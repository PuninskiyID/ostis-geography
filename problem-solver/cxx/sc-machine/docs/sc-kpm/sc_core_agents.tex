\scnheader{Реализация базового набора платформенно-зависимых sc-агентов и их общих компонентов}
\scnidtf{sc-kpm}
\begin{scnrelfromlist}{компонент программной системы}
    \scnitem{Реализация базового набора поисковых sc-агентов}
    \begin{scnindent}
        \begin{scnrelfromlist}{используемый язык программирования}
            \scnitem{С}
        \end{scnrelfromlist}
        \begin{scnrelfromlist}{компонент программной системы}
            \scnitem{Реализация Абстрактного sc-агента поиска семантической окрестности заданной сущности}
            \scnitem{Реализация Абстрактного sc-агента поиска всех сущностей, частных по отношению к заданной}
            \scnitem{Реализация Абстрактного sc-агента поиска всех сущностей, общих по отношению к заданной}
            \scnitem{Реализация Абстрактного sc-агента поиска всех sc-идентификаторов, соответствующих заданной сущности}
            \scnitem{Реализация Абстрактного sc-агента поиска базовых sc-дуг, инцидентных заданному sc-элементу}
            \begin{scnindent}
                \begin{scnrelfromlist}{компонент программной системы}
                    \scnitem{Реализация Абстрактного sc-агента поиска базовых sc-дуг, входящих в заданный sc-элемент}
                    \scnitem{Реализация Абстрактного sc-агента поиска базовых sc-дуг, выходящих из заданного sc-элемента}
                    \scnitem{Реализация Абстрактного sc-агента поиска базовых sc-дуг, входящих в заданный sc-элемент, с указанием множеств, которым принадлежат эти sc-дуги}
                    \scnitem{Реализация Абстрактного sc-агента поиска базовых sc-дуг, выходящих из заданного sc-элемента, с указанием множеств, которым принадлежат эти sc-дуги}
                \end{scnrelfromlist}
            \end{scnindent}
        \end{scnrelfromlist}
    \end{scnindent}
    \scnitem{Реализация базового механизма сборки информационного мусора}
    \begin{scnindent}
        \begin{scnrelfromlist}{используемый язык программирования}
            \scnitem{С}
        \end{scnrelfromlist}
        \scntext{примечание}{Текущая реализация механизма сборки информационного мусора содержит один sc-агент, реагирующий на явное добавление какого-либо sc-элемента во множество \scnqqi{информационный мусор} и осуществляющий физическое удаление этого sc-элемента из sc-памяти}
    \end{scnindent}
    \scnitem{Реализация базового набора интерфейсных sc-агентов}
    \begin{scnindent}
        \begin{scnrelfromlist}{используемый язык программирования}
            \scnitem{С++}
        \end{scnrelfromlist}
        \begin{scnrelfromlist}{компонент программной системы}
            \scnitem{Реализация Абстрактного sc-агента обработки команд пользовательского интерфейса}
            \scnitem{Реализация Абстрактного sc-агента трансляции из внутреннего представления знаний во промежуточный транспортный формат}
            \begin{scnindent}
                \scntext{примечание}{В настоящее время используется подход, при котором независимо от формы внешнего представления информации, информация хранимая в sc-памяти вначале транслируется в промежуточный транспортный формат на базе JSON, который затем обрабатывается sc-агентами пользовательского интерфейса, входящими в состав \textit{Реализации интерпретатора sc-моделей пользовательских интерфейсов}}
            \end{scnindent}
        \end{scnrelfromlist}
    \end{scnindent}
\end{scnrelfromlist}
